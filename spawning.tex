\clause{Task statements}
\sclause{Introduction}

\pnum
The grammar of a statement (6.8, paragraph 1)
\begin{cpp}
(clause 6, paragraph 1)
\end{cpp}
is modified to add task-statement as a new alternative.

\ssclause*{Syntax}

\begin{bnf}
\nontermdef{task-statement}
\br
task-block-statement
\br
task-spawn-statement
\br
task-sync-statement
\br
task-call-statement
\end{bnf}

\sclause{The task block statement}
\ssclause*{Syntax}

\begin{bnf}
\nontermdef{task-block-statement}
\br
\terminal{_Task_parallel} \terminal{_Block} compound-statement
\end{bnf}

\ssclause*{Semantics}

\pnum
Defines a task block, within which tasks can be spawned.
At the end of the contained compound statement,
all child tasks spawned directly or indirectly
within the compound statement are
\defn{joined}:
the execution of the following statement is dependent
on all tasks spawned within the compound statement; i.e.,
the statement following the task block is not executed
until all of the child tasks complete.
\footnote{TODO:
Is a branch allowed into or out of a task block?
}
\footnote{TODO:
What about C++ EH?
}

\pnum
For a given statement, the
\defn{associated task block}
is defined as follows.
For a statement within a task spawn statement,
there is no associated task block,
except within a nested task block statement
or parallel loop.
For a statement within a task block statement
or parallel loop,
the associated task block is the smallest enclosing task block statement
or parallel loop.
Otherwise, for a statement within the body of a function
declared with the spawning function specifier,
the associated task block is the same as it was
at the point of the task spawning call statement
that invoked the spawning function.
For a statement in any other context,
there is no associated task block.

\begin{note}
Task blocks can be nested lexically and/or dynamically.
Determination of the associated task block is a hybrid process:
lexically within a function,
and dynamically across calls to spawning functions.%
\footnote{DFEP:
In Cilk, this determination can be done entirely lexically.
In OpenMP, this determination can be done entirely dynamically.
}
Code designated for execution in another thread
by means other than a task statement
(e.g. using
\tcode{thrd_create})
is not part of any task block.
\end{note}

\sclause{The task spawn statement}
\ssclause*{Syntax}

\begin{bnf}
\nontermdef{task-spawn-statement}
\br
\terminal{_Task_parallel} \terminal{_Spawn} compound-statement
\end{bnf}

\ssclause*{Constraints}

\pnum
A task spawn statement shall have an associated task block.
\footnote{TODO:
Is a branch allowed into or out of a spawned statement?
}

\ssclause*{Semantics}
\pnum
The contained compound statement is executed as a task,
independent of the continued execution
of the associated task block.

\sclause{The task sync statement}
\ssclause*{Syntax}

\begin{bnf}
\nontermdef{task-sync-statement}
\br
\terminal{_Task_parallel} \terminal{_Sync} \terminal{;}
\end{bnf}

\ssclause*{Constraints}

\pnum
A task sync statement shall have an associated task block.

\ssclause*{Semantics}
\pnum
Joins all child tasks of the associated task block
of the task sync statement.

\sclause{The task spawning call statement}
\ssclause*{Syntax}
\begin{bnf}
\nontermdef{task-call-statement}
\br
\terminal{_Task_parallel} \terminal{_Call} expression-statement
\end{bnf}
\ssclause*{Constraints}

\pnum
A task spawning call statement shall have an associated task block.
\footnote{TODO:
Using the same keyword pair
for the statement prefix and the function specifier
introduces a case where three tokens of lookahead
(and hopefully no more)
are sometimes needed
to disambiguate a declaration from a statement.
Is this what we want?
}

\ssclause*{Semantics}

\pnum
The contained expression statement is executed normally.
Any called spawning function is allowed to spawn tasks;
any such tasks are associated with the associated task block
of the task spawning call statement,
and are
independent of
the statements of the task block
following the task spawning call statement.

\begin{note}
A call to a task spawning function need not be
the ``outermost'' operation of the expression statement.
A task spawning call statement
might invoke more than one spawning function,
or might invoke none.
\end{note}

\sclause{The spawning function specifier}
\ssclause*{Syntax}
\pnum
A new alternative is added to the grammar of function specifier
(6.7.4 paragraph 1):

\begin{bnf}
\nontermdef{function-specifier}
\br
\terminal{_Task_parallel} \terminal{_Call}
\end{bnf}

\ssclause*{Constraints}
\pnum
If a spawning function specifier appears
on any declaration of a function,
it shall appear on every declaration of that function.
A function declared with a spawning function specifier
shall be called only from a task spawning call statement.
\footnote{TODO:
Is a spawning function specifier part of the function's type?
If so, are function-pointer conversions allowed?
}