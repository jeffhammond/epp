\clause{Counted loops}
\sclause{Introduction}
A
\defn{counted loop}
is a
\tcode{for}
statement
\begin{cpp}
or range-based
\tcode{for}
statement
\end{cpp}
that is required to satisfy additional constraints.
The purpose of these constraints is to ensure that
the loop's iteration count can be computed
before the loop body is executed.

There shall be no
\tcode{return},
\tcode{break},
\tcode{goto}
or
\tcode{switch}
statement that might transfer control into or out of a counted loop.

\sclause{Constraints on a counted \tcode{for} statement}
\ssclause{Introduction}
The syntax of a
\tcode{for}
statement includes three
\defn{control clauses}
between parentheses,
separated by semicolons.
The first of these is called the initialization clause;
the second is called the condition clause or controlling expression;
the third is called the
\defn{loop-increment}.

When a constraint limits the form of an expression,
parentheses are allowed around the expression
or any required subexpression.

\ssclause{Constraints on the form of the control clauses}

\begin{cpp}
The
\nonterminal{condition}
shall be an expression.

\begin{note}
A condition with declaration form is useful
in a context where a value carries more information
than just whether it is zero or nonzero.
This is not believed to be useful in a counted loop.
\end{note}
\end{cpp}

The controlling expression shall be a comparison expression
with one of the following forms:%
\footnote{DFEP:
OpenMP does not (yet) allow comparison with
\tcode{!=}.
}

\begin{bnf}
\br
relational-expression \terminal{<} shift-expression
\br
relational-expression \terminal{>} shift-expression
\br
relational-expression \terminal{<=} shift-expression
\br
relational-expression \terminal{>=} shift-expression
\br
equality-expression \terminal{!=} relational-expression
\end{bnf}

Exactly one of the operands of the comparison operator
shall be an identifier designating an induction variable,
as described below.
This induction variable is known as the
\defn{control variable}.
The operand that is not the control variable is called the
\defn{limit expression}.
\begin{cpp}
Any implicit conversion applied to that operand is not considered
part of the limit expression.
\end{cpp}

The loop-increment shall be an expression with the following form:%
\footnote{DFEP:
OpenMP and ``classic'' Cilk allow only a single induction variable:
the loop control variable.
Allowing multiple induction variables is implemented in Intel's compiler.
}

\begin{bnf}
\nontermdef{loop-increment}
\br
single-increment
\br
loop-increment \terminal{,} single-increment
\end{bnf}

\begin{bnf}
\nontermdef{single-increment}
\br
identifier \terminal{++}
\br
identifier \terminal{--}
\br
\terminal{++} identifier
\br
\terminal{--} identifier
\br
identifier \terminal{+=} initializer-clause
\br
identifier \terminal{-=} initializer-clause
\br
identifier \terminal{=} identifier \terminal{+} multiplicative-expression
\br
identifier \terminal{=} identifier \terminal{-} multiplicative-expression
\br
identifier \terminal{=} additive-expression \terminal{+} identifier
\end{bnf}

\begin{cpp}
Each comma in the grammar of loop-increment shall represent
a use of the built-in comma operator.
\end{cpp}
The identifier in each grammatical alternative for single-increment
names an
\defn{induction variable}.
If
\grammarterm{identifier}
occurs twice in a grammatical alternative for
\grammarterm{single-increment},
the same variable shall be named by both occurrences.
If a grammatical alternative for
\grammarterm{single-increment}
contains a subexpression
that is not an identifier for the induction variable,
that is called the
\defn{stride expression}
for that induction variable.

An induction variable shall not be designated by more than one
\grammarterm{single-increment}.

\begin{note}
The control variable is identified by considering
the loop's condition and loop-increment together.
If exactly one operand of the condition comparison is a variable,
it is the control variable, and must be incremented.
If both operands of the condition comparison are variables,
only one is allowed to be incremented;
that one is the control variable.
It is an error if neither operand of the condition comparison is a variable.
\end{note}

\begin{note}
There is no additional constraint on the form
of the initialization clause of a counted \tcode{for} loop.%
\footnote{DFEP:
OpenMP and ``classic'' Cilk require that the control variable be initialized.
This relaxation is implemented in Intel's compiler.
}
\end{note}

\ssclause{Other statically checkable constraints}

Each induction variable shall have unqualified integral%
\yescpp{,}
\begin{cpp}
enumeration,
copy-constructible class,
\end{cpp}
or pointer type,
and shall have automatic storage duration.

Each stride expression shall have integral
\begin{cpp}
or enumeration
\end{cpp}
type.

\newcommand{\MYin}{\hspace{0.1in}}
\newcommand{\MYcolA}{0.7in}
\newcommand{\MYcolB}{1.0in}
\newcommand{\MYcolC}{1.3in}

\begin{table}[ht]
\caption{
Method of computing the iteration count
}
\label{tab:itcount}
\centering
\begin{tabular}{|l|l|l|l|l|}
\hline
\parbox[c][30pt]{\MYcolA}{
\bfseries
Form of\\condition
}&
\multicolumn{4}{c|}{
\bfseries
Form of single-increment
}
\\ \hline &
\parbox{\MYcolB}{
\bfseries
\grammarterm{id} \tcode{++}\\
\tcode{++} \grammarterm{id}
}&
\parbox{\MYcolB}{
\bfseries
\grammarterm{id} \tcode{--}\\
\tcode{--} \grammarterm{id}
}&
\parbox[c][40pt]{\MYcolC}{
\bfseries
\grammarterm{id} \tcode{+=} \grammarterm{stride}\\
\grammarterm{id} \tcode{=} \grammarterm{id} \tcode{+} \grammarterm{stride}\\
\grammarterm{id} \tcode{=} \grammarterm{stride}\tcode{+} \grammarterm{id}
}&
\parbox{\MYcolC}{
\bfseries
\grammarterm{id} \tcode{-=} \grammarterm{stride}\\
\grammarterm{id} \tcode{=} \grammarterm{id} \tcode{-} \grammarterm{stride}
}
\\ \hline
\parbox[c][30pt]{\MYcolA}{
\bfseries
\grammarterm{id} \tcode{<} \grammarterm{lim}\\
\grammarterm{lim} \tcode{>} \grammarterm{id}
}&
$((lim)-(id))$&
ERROR&
\parbox{\MYcolC}{
$((lim)-(id)-1)/$\\
\MYin$(stride)+1$
}&
\parbox{\MYcolC}{
$((lim)-(id)-1)/$\\
\MYin$(stride)+1$
}
\\ \hline
\parbox[c][30pt]{\MYcolA}{
\bfseries
\grammarterm{id} \tcode{>} \grammarterm{lim}\\
\grammarterm{lim} \tcode{<} \grammarterm{id}
}&
ERROR&
$((id)-(lim))$&
\parbox{\MYcolC}{
$((id)-(lim)-1)/$\\
\MYin$-(stride)+1$
}&
\parbox{\MYcolC}{
$((id)-(lim)-1)/$\\
\MYin$-(stride)+1$
}
\\ \hline
\parbox[c][30pt]{\MYcolA}{
\bfseries
\grammarterm{id} \tcode{<=} \grammarterm{lim}\\
\grammarterm{lim} \tcode{>=} \grammarterm{id}
}&
\parbox{1in}{
$((lim)-(id))$\\
\MYin+1
}&
ERROR&
\parbox{\MYcolC}{
$((lim)-(id))/$\\
\MYin$(stride)+1$
}&
\parbox{\MYcolC}{
$((lim)-(id))/$\\
\MYin$(stride)+1$
}
\\ \hline
\parbox[c][30pt]{\MYcolA}{
\bfseries
\grammarterm{id} \tcode{>=} \grammarterm{lim}\\
\grammarterm{lim} \tcode{<=} \grammarterm{id}
}&
ERROR&
\parbox{\MYcolB}{
$((id)-(lim))$\\
\MYin$+1$
}&
\parbox{\MYcolC}{
$((id)-(lim))/$\\
\MYin$-(stride)+1$
}&
\parbox{\MYcolC}{
$((id)-(lim))/$\\
\MYin$-(stride)+1$
}
\\ \hline
\parbox{\MYcolA}{
\bfseries
\grammarterm{id} \tcode{!=} \grammarterm{lim}\\
\grammarterm{lim} \tcode{!=} \grammarterm{id}
}&
$((lim)-(id)$&
$((id)-(lim))$&
\parbox[c][72pt]{\MYcolC}{
$((stride)<0)$ \tcode{?}\\
\MYin$((id)-(lim)-1)/$\\
\MYin\MYin$-(stride)+1$ \tcode{:}\\
\MYin$((lim)-(id)-1)/$\\
\MYin\MYin$(stride)+1$
}&
\parbox{\MYcolC}{
$((stride)<0)$ \tcode{?}\\
\MYin$((lim)-(id)-1)/$\\
\MYin\MYin$-(stride)+1$ \tcode{:}\\
\MYin$((id)-(lim)-1)/$\\
\MYin\MYin$(stride)+1$
}
\\ \hline
\end{tabular}
{\bfseries
Legend:
}

\begin{tabular}{|l|l|l|}
\hline
\bfseries Name&
\bfseries In the form of an expression&
\bfseries In the iteration count expression
\\ \hline
$id$&
The name of the control variable.&
\parbox[c][30pt]{2.5in}{
An expression with the type and value\\
of the control variable.
}
\\ \hline
$lim$&
The limit expression.&
\parbox[c][30pt]{2.5in}{
An expression with the type and value\\
of the limit expression.
}
\\ \hline
$stride$&
The stride expression.&
\parbox[c][40pt]{2.5in}{
An expression with the type and value\\
of the stride expression
for the control variable.
}
\\ \hline
\end{tabular}
\end{table}

The
\defn{iteration count}
is computed according to
\tref{tab:itcount}.%
\footnote{EN:
We need to say something about error cases.
}
The iteration count is computed after the loop initialization is performed,
and before the control variable is modified by the loop.
\begin{cpp}
The iteration count expression shall be well-formed.
\end{cpp}

The type of the difference between the limit expression and the control variable
is the
\defn{subtraction type}\nocpp{.}%
\yescpp{,}
\begin{cpp}
which shall be integral.
When the condition operation is !=,
(limit)-(var) and (var)-(limit) shall have the same type.
\end{cpp}
Each stride expression shall be convertible to the subtraction type.
\begin{cpp}
The loop odr-uses whatever operator-functions are selected
to compute these differences.
\end{cpp}

\begin{cpp}

\begin{table}[ht]
\caption{
Method of advancing an induction variable
}
\label{tab:inc}
\centering
\begin{tabular}{|l|l|}
\hline
\bfseries Single-increment operator&
\bfseries Expression
\\ \hline
\tcode{++ += +}&
$V$ \tcode{+=} $X$
\\ \hline
\tcode{-- -= -}&
$V$ \tcode{-=} $X$
\\ \hline
\end{tabular}
\end{table}

For each induction variable
$V$,
one of the expressions from
\tref{tab:inc}
shall be well-formed,
depending on the operator used in its single-increment.
In the table,
$X$
stands for some expression with the same type as the subtraction type.
The loop odr-uses whatever
\tcode{operator+=}
and
\tcode{operator-=}
functions are selected by these expressions.
\end{cpp}

\ssclause{Dynamic constraints}

If an induction variable is modified within the loop
other than as the side effect of its single-increment operation,
the behavior of the program is undefined.%

\begin{cpp}
If evaluation of the iteration count,
or a call to a required
\tcode{operator+=}
or
\tcode{operator-=}
function,
terminates with an exception,
the behavior of the program is undefined.
\end{cpp}

If $X$ and $Y$ are values of the control variable
that occur in consecutive evaluations of the loop condition
in the serialization,%
\footnote{EN:
We do not have a definition of ``serialization''
that is independent of syntax.
}
then the behavior is undefined if
$((limit) - X) - ((limit) - Y)$,
evaluated in infinite integer precision,
does not equal the stride.

\begin{note}
In other words, the control variable must obey the rules of normal arithmetic.
Unsigned wraparound is not allowed.
\end{note}

If the condition expression is true on entry to the loop,
then the behavior is undefined
if the computed iteration count is not greater than zero.
If the computed iteration count is not representable
as a value of type
\tcode{unsigned long long},
the behavior is undefined.

\ssclause{Evaluation relaxations}

The stride expressions shall not be evaluated if the iteration count is zero;
otherwise,
the stride and limit expressions are evaluated exactly once.%
\footnote{DFEP:
Neither OpenMP nor Cilk specifies
how many times these expressions must be evaluated.
}
%If execution of a loop iteration alters the value
%of the increment or limit expression,
%the behavior is undefined.

Within each iteration of the loop body,
the name of each induction variable refers to a local object,
as if the name were declared as an object within the body of the loop,
with automatic storage duration and with the type of the original object.
\begin{cpp}
If the loop body throws an exception
that is not caught within the same iteration of the loop,
the behavior is undefined, unless otherwise specified.
\end{cpp}

\begin{cpp}
\sclause{Constraints on a counted range-based \tcode{for} statement}
In a counted range-based
\tcode{for}
statement ([stmt.ranged] 6.5.4),
the type of the
\tcode{__begin}
variable,
as determined from the
\nonterminal{begin-expr},
shall satisfy the requirements of a random access iterator.
\begin{note}
Intel has not yet implemented support for
a parallel range-based
\tcode{for}
statement.
\end{note}
\end{cpp}