\clause{Parallel loop hint parameters \tcode{\textless{}cplex.h\textgreater{}}}

\sclause{Introduction}

\pnum
The header
\tcode{\textless cplex.h\textgreater}
defines several types and several macros.

\pnum
The
\pfxdefn{loop_params_t}
type is a structure type
with an unspecified number of members
for specifying parameters for tuning hints for a parallel loop.
A program whose output
depends on the value specified for any tuning hint parameter
is not considered a correct program.

\begin{note}
There is no guarantee that setting any tuning hint parameter
will improve the performance of the program.
\end{note}

\pnum
The
\pfxdefn{sched_kind_t}
type is an enumerated type
with at least the following enumeration constants,
each with nonzero value:

\begin{ttfamily}
\pfxdefn{sched_static}\\
\pfxdefn{sched_dynamic}\\
\pfxdefn{sched_guided}
\end{ttfamily}

\pnum
The
\pfxdefn{workload_t}
type is an enumerated type
with at least the following enumeration constants,
each with nonzero value:

\begin{ttfamily}
\pfxdefn{workload_balanced}\\
\pfxdefn{workload_unbalanced}
\end{ttfamily}

\pnum
The
\pfxdefn{affinity_t}
type is an enumerated type
with at least the following enumeration constants,
each with nonzero value:

\begin{ttfamily}
\pfxdefn{affinity_close}\\
\pfxdefn{affinity_spread}
\end{ttfamily}

\pnum
When an object of type
\pfx{loop_params_t}
is used as the loop parameter of a parallel loop,
the loop is described as being associated with the object.%
\footnote{EN:
What if the object is modified during the execution of the loop?
}
When executing a parallel loop associated with an object of type
\pfx{loop_params_t},
for any parameter for which the corresponding member has the value zero,
an unspecified default value is used.

\pnum
Each parameter is represented by a pair of macros:
one to set the value of the parameter in the parameter block,
and one to get the value of the parameter from the parameter block.

\begin{note}
Because these methods are specified as macros,
not functions,
taking the address of any of them need not be supported.
However, an implementation is also free
to provide functions with these names.
\end{note}

\begin{example}
Hint parameters for a parallel loop can be specified as follows:
\begin{verbatim}
#include <cplex.h>
#include <stdlib.h>
int main(int argc, char *argv[])
{
    cplex_loop_params_t hints = { 0 };
    if (argc > 1) {
        cplex_set_num_threads(&hints, atoi(argv[1]));
    }
    cplex_set_chunk_size(&hints, 1000);
    for _Task_parallel[hints] (long i = 0; i < 1000000; i++) {
        do_something_with(i);
    }
}
\end{verbatim}
\end{example}

\sclause{The \tcode{num_threads} parameter}
\ssclause*{Synopsis}
\begin{ttfamily}
\#include <cplex.h>\\
void \pfxdefn{set_num_threads}(\pfx{loop_params_t} *hints, int num_threads);\\
int \pfx{get_num_threads}(\pfx{loop_params_t} *hints);
\end{ttfamily}

\ssclause*{Description}
\pnum
The
\pfx{set_num_threads}
macro sets to
\tcode{num_threads}
the recommended number of execution agents to be used
to execute the iterations of a parallel loop
associated with the object pointed to by
\tcode{hints}.

\sclause{The \tcode{chunk_size} parameter}
\ssclause*{Synopsis}
\begin{ttfamily}
\#include <cplex.h>\\
void \pfxdefn{set_chunk_size}(\pfx{loop_params_t} *hints, int chunk_size);\\
int \pfx{get_chunk_size}(\pfx{loop_params_t} *hints);
\end{ttfamily}

\ssclause*{Description}
\pnum
The
\pfx{set_chunk_size}
macro sets to
\tcode{chunk_size}
the recommended maximum number of iterations
of a parallel loop associated with the object pointed to by
\tcode{hints}
to be grouped together to be executed sequentially
in a single thread of execution.

\sclause{The \tcode{schedule_kind} parameter}
\ssclause*{Synopsis}
\begin{ttfamily}
\#include <cplex.h>\\
void \pfxdefn{set_schedule_kind}(\pfx{loop_params_t} *hints, \pfx{sched_kind_t} kind);\\
\pfx{sched_kind_t} \pfx{get_schedule_kind}(\pfx{loop_params_t} *hints);
\end{ttfamily}

\ssclause*{Description}
\pnum
The
\pfx{set_schedule_kind}
macro sets to
\tcode{kind}
the recommended scheduling algorithm
for a parallel loop associated with the object pointed to by
\tcode{hints}.

\begin{note}
Setting the
\tcode{schedule_kind}
parameter to a particular value
may (but need not)
select the corresponding OpenMP loop-scheduling algorithm.
\end{note}

\sclause{The \tcode{workload_balance} parameter}
\ssclause*{Synopsis}
\begin{ttfamily}
\#include <cplex.h>\\
void \pfxdefn{set_workload_balance}(\pfx{loop_params_t} *hints, \pfx{workload_t} kind);\\
\pfx{workload_t} \pfx{get_workload_balance}(\pfx{loop_params_t} *hints);
\end{ttfamily}

\ssclause*{Description}
\pnum
The
\pfx{set_workload_balance}
macro sets to
\tcode{kind}
the workload-balancing characteristic
for a parallel loop associated with the object pointed to by
\tcode{hints}.

\pnum
For a loop with a balanced workload,
each iteration should be assumed to execute
in approximately the same amount of time.
A loop with an unbalanced workload
should be assumed to have iterations
taking widely varying amounts of time.

\begin{note}
This parameter is semantically a statement about the associated loop,
whereas the
\tcode{schedule_kind}
parameter is semantically a request to the implementation.
Setting this parameter to
\pfx{workload_balanced}
may have an effect similar to setting the schedule to
\pfx{schedule_static}.
Setting this parameter to
\pfx{workload_unbalanced}
may have an effect similar to setting the schedule to
\pfx{schedule_dynamic}
or
\pfx{schedule_guided}.
\end{note}

\sclause{The \tcode{affinity} parameter}
\ssclause*{Synopsis}
\begin{ttfamily}
\#include <cplex.h>\\
void \pfxdefn{set_affinity}(\pfx{loop_params_t} *hints, \pfx{affinity_t} kind);\\
\pfx{affinity_t} \pfx{get_affinity}(\pfx{loop_params_t} *hints);
\end{ttfamily}

\ssclause*{Description}
\pnum
The
\pfx{set_affinity}
macro sets to
\tcode{kind}
the recommended affinity
for a parallel loop associated with the object pointed to by
\tcode{hints}.

\pnum
The affinity of a loop indicates whether the loop benefits
from being executed by co-located hardware threads,
or whether performance is likely to improve if the software threads
are spread over multiple cores.
