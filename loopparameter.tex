\clause{Parallel loop hint parameters \tcode{\textless{}cplex.h\textgreater{}}}

\sclause{Introduction}

\pnum
The header
\tcode{\textless cplex.h\textgreater}
defines two types and several macros.
The types are
\begin{verbatim}
cplex_loop_params_t
\end{verbatim}
which is a structure type
with an unspecified number of members
for specifying parameters for tuning hints for a parallel loop,
and
\begin{verbatim}
cplex_sched_kind_t
\end{verbatim}
which is an enumerated type
with at least the following enumeration constants,
each with nonzero value:
\begin{verbatim}
cplex_sched_static
cplex_sched_dynamic
cplex_sched_guided
\end{verbatim}

\begin{note}
Using one of these constants as the value of the
\tcode{kind}
parameter to
\tcode{cplex_set_schedule_kind}
may (but need not)
select the corresponding OpenMP loop-scheduling algorithm.
\end{note}

\pnum
When an object of type
\tcode{cplex_loop_params_t}
is used as the loop parameter of a parallel loop,
the loop is described as being associated with the object.%
\footnote{EN:
What if the object is modified during the execution of the loop?
}
When executing a parallel loop associated with an object of type
\tcode{cplex_loop_params_t},
for any parameter for which the corresponding member has the value zero,
an unspecified default value is used.

\begin{example}
Hint parameters for a parallel loop can be specified as follows:
\begin{verbatim}
#include <cplex.h>
#include <stdlib.h>
int main(int argc, char *argv[])
{
    cplex_loop_params_t hints = { 0 };
    if (argc > 1) {
        cplex_set_num_threads(&hints, atoi(argv[1]));
    }
    cplex_set_chunk_size(&hints, 1000);
    for _Task_parallel[hints] (long i = 0; i < 1000000; i++) {
        do_something_with(i);
    }
}
\end{verbatim}
\end{example}

\begin{note}
Because the following methods are specified as macros,
not functions,
taking the address of any of them need not be supported.
However, an implementation is also free
to provide functions with these names.
\end{note}

\sclause{The \tcode{cplex_set_num_threads} macro}
\ssclause*{Synopsis}
\begin{verbatim}
#include <cplex.h>
void cplex_set_num_threads(cplex_loop_params_t *hints, int num_threads);
\end{verbatim}

\ssclause*{Description}
\pnum
Sets to
\tcode{num_threads}
the recommended number of execution agents to be used
to execute the iterations of a parallel loop
associated with the object pointed to by
\tcode{hints}.

\sclause{The \tcode{cplex_get_num_threads} macro}
\ssclause*{Synopsis}
\begin{verbatim}
#include <cplex.h>
int cplex_get_num_threads(cplex_loop_params_t *hints);
\end{verbatim}

\ssclause*{Description}
\pnum
Retrieves the value set by
\tcode{cplex_set_num_threads}
from the object pointed to by
\tcode{hints}.

\sclause{The \tcode{cplex_set_chunk_size} macro}
\ssclause*{Synopsis}
\begin{verbatim}
#include <cplex.h>
void cplex_set_chunk_size(cplex_loop_params_t *hints, int chunk_size);
\end{verbatim}

\ssclause*{Description}
\pnum
Sets to
\tcode{chunk_size}
the recommended maximum number of iterations
of a parallel loop associated with the object pointed to by
\tcode{hints}
to be grouped together to be executed sequentially
in a single thread of execution.

\sclause{The \tcode{cplex_get_chunk_size} macro}
\ssclause*{Synopsis}
\begin{verbatim}
#include <cplex.h>
int cplex_get_chunk_size(cplex_loop_params_t *hints);
\end{verbatim}

\ssclause*{Description}
\pnum
Retrieves the value set by
\tcode{cplex_set_chunk_size}
from the object pointed to by
\tcode{hints}.

\sclause{The \tcode{cplex_set_schedule_kind} macro}
\ssclause*{Synopsis}
\begin{verbatim}
#include <cplex.h>
void cplex_set_schedule_kind(cplex_loop_params_t *hints, cplex_sched_kind_t kind);
\end{verbatim}

\ssclause*{Description}
\pnum
Sets to
\tcode{kind}
the recommended scheduling algorithm
for a parallel loop associated with the object pointed to by
\tcode{hints}.

\sclause{The \tcode{cplex_get_schedule_kind} macro}
\ssclause*{Synopsis}
\begin{verbatim}
#include <cplex.h>
cplex_sched_kind_t cplex_get_schedule_kind(cplex_loop_params_t *hints);
\end{verbatim}

\ssclause*{Description}
\pnum
Retrieves the value set by
\tcode{cplex_set_schedule_kind}
from the object pointed to by
\tcode{hints}.