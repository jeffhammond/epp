% Use of LuaTeX seems to be recommended.
% This stuff is all about "standard" LaTex capabilities used in the document.
\documentclass[
	techspec,	% for a Technical Specification
	final,		% needed for PDF bookmarks
	notcopyright,	% no copyright statements (yet)
	letterpaper	% for now
	]{isov2}
\let\ifpdf\relaxed	% let hyperref implement \ifpdf
\usepackage[
	pdftitle={Extensions for parallel programming},
	pdfauthor={Clark Nelson},
	pdfsubject={WORKING DRAFT Technical Specification},
	pdfkeywords={},
	pdfstartview=FitH,
	colorlinks=true, % not boxed
	linkcolor=blue,	% usual default for browsers
	urlcolor=blue
	]{hyperref}	% for PDF hyperlinks and bookmarks
\usepackage{chngcntr}	% for paragraph numbering
\usepackage{color}	% for editorial marking
\usepackage[T1]{fontenc} % to use 256-character fonts
\usepackage{fontspec}	% to disable ligatures
\usepackage{ifthen} 	% for conditional code/text
\usepackage{listings}	% for code listings
\usepackage{makeidx}	% for indexing
\usepackage[normalem]{ulem}	% for underlining and strikeout
\usepackage{underscore} % to simplify use of underscores
\usepackage{verbatim}	% for code font
\usepackage{xspace}	% to simplify macros that expand to text
\makeindex
\setcounter{secnumdepth}{6}
\setcounter{tocdepth}{2} % practical TOC size
\setcounter{tocdepth}{6} % uncomment to check outline structure

% These macros are from the WG21 standard source.
% Definitions and redefinitions of special commands

%%--------------------------------------------------
%% Difference markups
\definecolor{addclr}{rgb}{0,.6,.6}
\definecolor{remclr}{rgb}{1,0,0}
\definecolor{noteclr}{rgb}{0,0,1}

\renewcommand{\added}[1]{\textcolor{addclr}{\uline{#1}}}
\newcommand{\removed}[1]{\textcolor{remclr}{\sout{#1}}}
\begin{comment}
\renewcommand{\changed}[2]{\removed{#1}\added{#2}}
\end{comment}
\newcommand{\changed}[2]{\removed{#1}\added{#2}}

\newcommand{\nbc}[1]{[#1]\ }
\newcommand{\addednb}[2]{\added{\nbc{#1}#2}}
\newcommand{\removednb}[2]{\removed{\nbc{#1}#2}}
\newcommand{\changednb}[3]{\removednb{#1}{#2}\added{#3}}
\newcommand{\remitem}[1]{\item\removed{#1}}

\newcommand{\ednote}[1]{\textcolor{noteclr}{[Editor's note: #1] }}
% \newcommand{\ednote}[1]{}

\newenvironment{addedblock}
{
\color{addclr}
}
{
\color{black}
}
\newenvironment{removedblock}
{
\color{remclr}
}
{
\color{black}
}

%%--------------------------------------------------
%% Sectioning macros.  
% Each section has a depth, an automatically generated section 
% number, a name, and a short tag.  The depth is an integer in 
% the range [0,5].  (If it proves necessary, it wouldn't take much
% programming to raise the limit from 5 to something larger.)


% The basic sectioning command.  Example:
%    \Sec1[intro.scope]{Scope}
% defines a first-level section whose name is "Scope" and whose short
% tag is intro.scope.  The square brackets are mandatory.
\def\Sec#1[#2]#3{{%
\ifcase#1\let\s=\chapter
      \or\let\s=\section
      \or\let\s=\subsection
      \or\let\s=\subsubsection
      \or\let\s=\paragraph
      \or\let\s=\subparagraph
      \fi%
\s[#3]{#3\hfill[#2]}\label{#2}}}

% A convenience feature (mostly for the convenience of the Project
% Editor, to make it easy to move around large blocks of text):
% the \rSec macro is just like the \Sec macro, except that depths 
% relative to a global variable, SectionDepthBase.  So, for example,
% if SectionDepthBase is 1,
%   \rSec1[temp.arg.type]{Template type arguments}
% is equivalent to
%   \Sec2[temp.arg.type]{Template type arguments}
\newcounter{SectionDepthBase}
\newcounter{scratch}

\def\rSec#1[#2]#3{{%
\setcounter{scratch}{#1}
\addtocounter{scratch}{\value{SectionDepthBase}}
\Sec{\arabic{scratch}}[#2]{#3}}}

\newcommand{\synopsis}[1]{\textbf{#1}}

%%--------------------------------------------------
% Indexing

% locations
\newcommand{\indextext}[1]{\index[generalindex]{#1}}
\newcommand{\indexlibrary}[1]{\index[libraryindex]{#1}}
\newcommand{\indexgram}[1]{\index[grammarindex]{#1}}
\newcommand{\indeximpldef}[1]{\index[impldefindex]{#1}}

\newcommand{\indexdefn}[1]{\indextext{#1}}
\newcommand{\indexgrammar}[1]{\indextext{#1}\indexgram{#1}}
\newcommand{\impldef}[1]{\indeximpldef{#1}implementation-defined}

% appearance
\newcommand{\idxcode}[1]{#1@\tcode{#1}}
\newcommand{\idxhdr}[1]{#1@\tcode{<#1>}}
\newcommand{\idxgram}[1]{#1@\textit{#1}}

%%--------------------------------------------------
% General code style
\newcommand{\CodeStyle}{\ttfamily}
\newcommand{\CodeStylex}[1]{\texttt{#1}}

% Code and definitions embedded in text.
\newcommand{\tcode}[1]{\CodeStylex{#1}}
\newcommand{\techterm}[1]{\textit{#1}\xspace}
\newcommand{\defnx}[2]{\indexdefn{#2}\textit{#1}\xspace}
\newcommand{\defn}[1]{\defnx{#1}{#1}}
\newcommand{\term}[1]{\textit{#1}\xspace}
\newcommand{\grammarterm}[1]{\textit{#1}\xspace}

%%--------------------------------------------------
%% allow line break if needed for justification
\newcommand{\brk}{\discretionary{}{}{}}
%  especially for scope qualifier
\newcommand{\colcol}{\brk::\brk}

%%--------------------------------------------------
%% Macros for funky text
\newcommand{\Rplus}{\protect\hspace{-.1em}\protect\raisebox{.35ex}{\smaller{\smaller\textbf{+}}}}
% \newcommand{\Rplus}{+}
\newcommand{\Cpp}{\mbox{C\Rplus\Rplus}\xspace}
\newcommand{\CppIII}{\Cpp 2003\xspace}
\newcommand{\CppXI}{\Cpp 2011\xspace}
\newcommand{\opt}{{\ensuremath{_\mathit{opt}}}\xspace}
\newcommand{\shl}{<{<}}
\newcommand{\shr}{>{>}}
\newcommand{\dcr}{-{-}}
\newcommand{\exor}{\^{}}
\newcommand{\bigoh}[1]{\ensuremath{\mathscr{O}(#1)}}
% \renewcommand{\tilde}{{\smaller\ensuremath{\sim}}}    % extra level of braces is necessary
% \renewcommand{\tilde}{\protect\raisebox{-.5ex}{\larger\textasciitilde}}

%%--------------------------------------------------
%% States and operators
\newcommand{\state}[2]{\tcode{#1}\ensuremath{_{#2}}}
\newcommand{\bitand}{\ensuremath{\, \mathsf{bitand} \,}}
\newcommand{\bitor}{\ensuremath{\, \mathsf{bitor} \,}}
\newcommand{\xor}{\ensuremath{\, \mathsf{xor} \,}}
\newcommand{\rightshift}{\ensuremath{\, \mathsf{rshift} \,}}
\newcommand{\leftshift}[1]{\ensuremath{\, \mathsf{lshift}_#1 \,}}

%% Notes and examples
\newcommand{\EnterBlock}[1]{[\,\textit{#1:}\xspace}
\newcommand{\ExitBlock}[1]{\textit{\,---\,end #1}\,]\xspace}
\newcommand{\enternote}{\EnterBlock{Note}}
\newcommand{\exitnote}{\ExitBlock{note}}
\newcommand{\enterexample}{\EnterBlock{Example}}
\newcommand{\exitexample}{\ExitBlock{example}}

%% Library function descriptions
\newcommand{\Fundescx}[1]{\textit{#1}\xspace}
\newcommand{\Fundesc}[1]{\Fundescx{#1:}}
\newcommand{\required}{\Fundesc{Required behavior}}
\newcommand{\requires}{\Fundesc{Requires}}
\newcommand{\effects}{\Fundesc{Effects}}
\newcommand{\postconditions}{\Fundesc{Postconditions}}
\newcommand{\postcondition}{\Fundesc{Postcondition}}
\newcommand{\preconditions}{\requires}
\newcommand{\precondition}{\requires}
\newcommand{\returns}{\Fundesc{Returns}}
\newcommand{\throws}{\Fundesc{Throws}}
\newcommand{\default}{\Fundesc{Default behavior}}
\newcommand{\complexity}{\Fundesc{Complexity}}
\newcommand{\remark}{\Fundesc{Remark}}
\newcommand{\remarks}{\Fundesc{Remarks}}
\begin{comment}
\newcommand{\note}{\remark}
\end{comment}
\newcommand{\notes}{\remarks}
\newcommand{\realnote}{\Fundesc{Note}}
\newcommand{\realnotes}{\Fundesc{Notes}}
\newcommand{\errors}{\Fundesc{Error conditions}}
\newcommand{\sync}{\Fundesc{Synchronization}}
\newcommand{\implimits}{\Fundesc{Implementation limits}}
\newcommand{\replaceable}{\Fundesc{Replaceable}}
\newcommand{\exceptionsafety}{\Fundesc{Exception safety}}
\newcommand{\returntype}{\Fundesc{Return type}}
\newcommand{\cvalue}{\Fundesc{Value}}
\newcommand{\ctype}{\Fundesc{Type}}
\newcommand{\ctypes}{\Fundesc{Types}}
\newcommand{\dtype}{\Fundesc{Default type}}
\newcommand{\ctemplate}{\Fundesc{Class template}}
\newcommand{\templalias}{\Fundesc{Alias template}}

%% Cross reference
\newcommand{\xref}{\textsc{See also:}\xspace}
\newcommand{\xsee}{\textsc{See:}\xspace}

%% NTBS, etc.
\newcommand{\NTS}[1]{\textsc{#1}\xspace}
\newcommand{\ntbs}{\NTS{ntbs}}
\newcommand{\ntmbs}{\NTS{ntmbs}}
\newcommand{\ntwcs}{\NTS{ntwcs}}
\newcommand{\ntcxvis}{\NTS{ntc16s}}
\newcommand{\ntcxxxiis}{\NTS{ntc32s}}

%% Code annotations
\newcommand{\EXPO}[1]{\textit{#1}}
\newcommand{\expos}{\EXPO{exposition only}}
\newcommand{\exposr}{\expos}
\newcommand{\exposrc}{\expos}
\newcommand{\impdef}{\EXPO{implementation-defined}}
\newcommand{\impdefx}[1]{\indeximpldef{#1}\EXPO{implementation-defined}}
\newcommand{\notdef}{\EXPO{not defined}}

\newcommand{\UNSP}[1]{\textit{\texttt{#1}}}
\newcommand{\unspec}{\UNSP{unspecified}\xspace}
\newcommand{\unspecbool}{\UNSP{unspecified-bool-type}}
\newcommand{\seebelow}{\UNSP{see below}}
\newcommand{\unspecuniqtype}{\UNSP{unspecified unique type}}
\newcommand{\unspecalloctype}{\UNSP{unspecified allocator type}}

%% Double underscore
\newcommand{\unun}{\_\,\_}
\newcommand{\xname}[1]{\unun\,#1}
\newcommand{\mname}[1]{\tcode{\unun\,#1\,\unun}}

%% Ranges
\newcommand{\Range}[4]{\tcode{#1\brk{}#3,\brk{}#4\brk{}#2}\xspace}
\newcommand{\crange}[2]{\Range{[}{]}{#1}{#2}}
\newcommand{\brange}[2]{\Range{(}{]}{#1}{#2}}
\newcommand{\orange}[2]{\Range{(}{)}{#1}{#2}}
\newcommand{\range}[2]{\Range{[}{)}{#1}{#2}}

%% Change descriptions
\newcommand{\diffdef}[1]{\hfill\break\textbf{#1:}\xspace}
\newcommand{\change}{\diffdef{Change}}
\newcommand{\rationale}{\diffdef{Rationale}}
\newcommand{\effect}{\diffdef{Effect on original feature}}
\newcommand{\difficulty}{\diffdef{Difficulty of converting}}
\newcommand{\howwide}{\diffdef{How widely used}}

%% Miscellaneous
\newcommand{\uniquens}{\textrm{\textit{\textbf{unique}}} }
\newcommand{\stage}[1]{\item{\textbf{Stage #1:}}\xspace}
\newcommand{\doccite}[1]{\textit{#1}\xspace}
\newcommand{\cvqual}[1]{\textit{#1}}
\newcommand{\cv}{\cvqual{cv}}
\renewcommand{\emph}[1]{\textit{#1}\xspace}
\newcommand{\numconst}[1]{\textsl{#1}\xspace}
\newcommand{\logop}[1]{{\footnotesize #1}\xspace}

%%--------------------------------------------------
%% Environments for code listings.

% We use the 'listings' package, with some small customizations.  The
% most interesting customization: all TeX commands are available
% within comments.  Comments are set in italics, keywords and strings
% don't get special treatment.

\lstset{language=C++,
        basicstyle=\small\CodeStyle,
        keywordstyle=,
        stringstyle=,
        xleftmargin=1em,
        showstringspaces=false,
        commentstyle=\itshape\rmfamily,
        columns=flexible,
        keepspaces=true,
        texcl=true}

% Our usual abbreviation for 'listings'.  Comments are in 
% italics.  Arbitrary TeX commands can be used if they're 
% surrounded by @ signs.
\lstnewenvironment{codeblock}
{
 \lstset{escapechar=@}
 \renewcommand{\tcode}[1]{\textup{\CodeStylex{##1}}}
 \renewcommand{\techterm}[1]{\textit{\CodeStylex{##1}}}
 \renewcommand{\term}[1]{\textit{##1}}
 \renewcommand{\grammarterm}[1]{\textit{##1}}
}
{
}

% Permit use of '@' inside codeblock blocks (don't ask)
\makeatletter
\newcommand{\atsign}{@}
\makeatother

%%--------------------------------------------------
%% Indented text
\newenvironment{indented}
{\list{}{}\item\relax}
{\endlist}

%%--------------------------------------------------
%% Library item descriptions
\lstnewenvironment{itemdecl}
{
 \lstset{escapechar=@,
 xleftmargin=0em,
 aboveskip=2ex,
 belowskip=0ex	% leave this alone: it keeps these things out of the
				% footnote area
 }
}
{
}

\newenvironment{itemdescr}
{
 \begin{indented}}
{
 \end{indented}
}


%%--------------------------------------------------
%% Bnf environments
\newlength{\BnfIndent}
\setlength{\BnfIndent}{\leftmargini}
\newlength{\BnfInc}
\setlength{\BnfInc}{\BnfIndent}
\newlength{\BnfRest}
\setlength{\BnfRest}{2\BnfIndent}
\newcommand{\BnfNontermshape}{\small\rmfamily\itshape}
\newcommand{\BnfTermshape}{\small\ttfamily\upshape}
\newcommand{\nonterminal}[1]{{\BnfNontermshape #1}}

\newenvironment{bnfbase}
 {
 \newcommand{\nontermdef}[1]{\indexgrammar{\idxgram{##1}}\nonterminal{##1}:}
 \newcommand{\terminal}[1]{{\BnfTermshape ##1}\xspace}
 \newcommand{\descr}[1]{\normalfont{##1}}
 \newcommand{\bnfindentfirst}{\BnfIndent}
 \newcommand{\bnfindentinc}{\BnfInc}
 \newcommand{\bnfindentrest}{\BnfRest}
 \begin{minipage}{.9\hsize}
 \newcommand{\br}{\hfill\\}
 }
 {
 \end{minipage}
 }

\newenvironment{BnfTabBase}[1]
{
 \begin{bnfbase}
 #1
 \begin{indented}
 \begin{tabbing}
 \hspace*{\bnfindentfirst}\=\hspace{\bnfindentinc}\=\hspace{.6in}\=\hspace{.6in}\=\hspace{.6in}\=\hspace{.6in}\=\hspace{.6in}\=\hspace{.6in}\=\hspace{.6in}\=\hspace{.6in}\=\hspace{.6in}\=\hspace{.6in}\=\kill}
{
 \end{tabbing}
 \end{indented}
 \end{bnfbase}
}

\newenvironment{bnfkeywordtab}
{
 \begin{BnfTabBase}{\BnfTermshape}
}
{
 \end{BnfTabBase}
}

\newenvironment{bnftab}
{
 \begin{BnfTabBase}{\BnfNontermshape}
}
{
 \end{BnfTabBase}
}

\newenvironment{simplebnf}
{
 \begin{bnfbase}
 \BnfNontermshape
 \begin{indented}
}
{
 \end{indented}
 \end{bnfbase}
}

\newenvironment{bnf}
{
 \begin{bnfbase}
 \list{}
	{
	\setlength{\leftmargin}{\bnfindentrest}
	\setlength{\listparindent}{-\bnfindentinc}
	\setlength{\itemindent}{\listparindent}
	}
 \BnfNontermshape
 \item\relax
}
{
 \endlist
 \end{bnfbase}
}

% non-copied versions of bnf environments
\newenvironment{ncbnftab}
{
 \begin{bnftab}
}
{
 \end{bnftab}
}

\newenvironment{ncsimplebnf}
{
 \begin{simplebnf}
}
{
 \end{simplebnf}
}

\newenvironment{ncbnf}
{
 \begin{bnf}
}
{
 \end{bnf}
}

%%--------------------------------------------------
%% Drawing environment
%
% usage: \begin{drawing}{UNITLENGTH}{WIDTH}{HEIGHT}{CAPTION}
\newenvironment{drawing}[4]
{
\newcommand{\mycaption}{#4}
\begin{figure}[h]
\setlength{\unitlength}{#1}
\begin{center}
\begin{picture}(#2,#3)\thicklines
}
{
\end{picture}
\end{center}
\caption{\mycaption}
\end{figure}
}

%%--------------------------------------------------
%% Environment for imported graphics
% usage: \begin{importgraphic}{CAPTION}{TAG}{FILE}

\newenvironment{importgraphic}[3]
{%
\newcommand{\cptn}{#1}
\newcommand{\lbl}{#2}
\begin{figure}[htp]\centering%
\includegraphics[scale=.35]{#3}
}
{
\caption{\cptn}\label{\lbl}%
\end{figure}}

%% enumeration display overrides
% enumerate with lowercase letters
\newenvironment{enumeratea}
{
 \renewcommand{\labelenumi}{\alph{enumi})}
 \begin{enumerate}
}
{
 \end{enumerate}
}

% enumerate with arabic numbers
\newenvironment{enumeraten}
{
 \renewcommand{\labelenumi}{\arabic{enumi})}
 \begin{enumerate}
}
{
 \end{enumerate}
}

%%--------------------------------------------------
%% Definitions section
% usage: \definition{name}{xref}
%\newcommand{\definition}[2]{\rSec2[#2]{#1}}
% for ISO format, use:
\begin{comment}
\newcommand{\definition}[2]
 {\hfill\vspace{.25ex plus .5ex minus .2ex}\\
 \addtocounter{subsection}{1}%
 \textbf{\thesubsection\hfill\relax[#2]}\\
 \textbf{#1}\label{#2}\\
 }
\end{comment}

\renewcommand{\indextext}[1]{\index{#1}}
\renewcommand{\indexgram}[1]{\index{#1}}

% This is for macros invented for/by CPLEX.
\newboolean{hidecpp}
\setboolean{hidecpp}{false}
\ifthenelse{\boolean{hidecpp}}
{\let\cpp=\comment
\let\endcpp=\endcomment
\newcommand{\nocpp}[1]{#1}
\newcommand{\yescpp}[1]{}
}
{\newenvironment{cpp}
{\sffamily\slshape[C++:}
{]}
\newcommand{\nocpp}[1]{}
\newcommand{\yescpp}[1]{#1}
}
%% Paragraph numbering
\newcounter{Paras}
\counterwithin{Paras}{clause}
\counterwithin{Paras}{sclause}
\counterwithin{Paras}{ssclause}
\counterwithin{Paras}{sssclause}
\counterwithin{Paras}{ssssclause}
\counterwithin{Paras}{sssssclause}
\makeatletter
\def\pnum{\addtocounter{Paras}{1}\noindent\llap{{%
  \scriptsize\raisebox{.7ex}{\arabic{Paras}}}\hspace{\@totalleftmargin}\quad}}
\renewcommand{\definition}[2]{\@defcl{#1}\index{#1} #2}
\makeatother


% These parameters, and the documentclass options,
% are about document identification and overall formatting.
\newcommand{\cplexts}{$CPLEXTS$}
\renewcommand{\extrahead}{2014-08-21}
\standard{\cplexts}
\yearofedition{2016}
\languageofedition{(E)}

\begin{document}

\begin{cover}
This is the cover page.
When ISO publishes a document, they apply their own title page,
so CPLEX/WG14 can put anything here.
\clearpage
\end{cover}

% The table of contents appears here.

\begin{foreword}
ISO (the International Organization for Standardization) is a worldwide federation of national standards
bodies (ISO member bodies). The work of preparing International Standards is normally carried out
through ISO technical committees. Each member body interested in a subject for which a technical
committee has been established has the right to be represented on that committee. International organizations,
governmental and non-governmental, in liaison with ISO, also take part in the work. ISO
collaborates closely with the International Electrotechnical Commission (IEC) on all matters of electrotechnical
standardization.

International Standards are drafted in accordance with the rules given in the ISO/IEC Directives, Part 2.

The main task of technical committees is to prepare International Standards. Draft International Standards
adopted by the technical committees are circulated to the member bodies for voting. Publication
as an International Standard requires approval by at least 75\% of the member bodies casting a vote.

In other circumstances, particularly when there is an urgent market requirement for such documents, a
technical committee may decide to publish other types of normative document:

\begin{itemize}
\item
an ISO Publicly Available Specification (ISO/PAS) represents an agreement between technical experts
in an ISO working group and is accepted for publication if it is approved by more than 50\%
of the members of the parent committee casting a vote;
\item
an ISO Technical Specification (ISO/TS) represents an agreement between the members of a technical
committee and is accepted for publication if it is approved by 2/3 of the members of the
committee casting a vote.
\end{itemize}

An ISO/PAS or ISO/TS is reviewed every three years with a view to deciding whether it can be transformed
into an International Standard.


ISO/TS
\cplexts{}
was prepared by Technical Committee ISO/IEC JTC1/SC22/WG14.%
\footnote{EN:
This is the only paragraph in the Foreword that has anything in it
that's not just boilerplate.
}

\fwdnopatents
\end{foreword}

\begin{introduction}
\intropatents
\end{introduction}

\title{Programming languages}{C}{Extensions for parallel programming}

\scopeclause
\begin{inscope}{technical specification}
\item
Extensions to the C language to simplify writing a parallel program.
\end{inscope}
\begin{outofscope}{technical specification}
\item
Support for writing a concurrent program.
\end{outofscope}

\normrefsclause
\normrefbp{technical specification}
\begin{nreferences}
\isref{ISO/IEC 9899:2011(E)}{Programming languages --- C}
\disref{ISO/IEC 14882:2014(E)}{Programming languages --- C++}
\end{nreferences}

\defclause
\begin{definitions}

For the purposes of this document,
the following terms and definitions apply.

\definition{thread of execution}{
flow of control within a program,
including a top-level statement or expression,
and recursively including every function invocation it executes%
\footnote{EN:
Adapted from the C++ standard.
}
}

\definition{OS thread}{
service provided by an operating system
for executing multiple threads of execution concurrently
}

\begin{note}
There is typically significant overhead involved
in creating a new OS thread.
\end{note}

\definition{thread}{
thread of execution, or OS thread
}

\begin{note}
This word, when used without qualification, is ambiguous.
\end{note}

\definition{execution agent}{
entity, such as an OS thread,
that may perform work in parallel with other execution agents%
\footnote{EN:
Adapted from the C++ standard.
}
}

\definition{task}{
thread of execution within a program
that can be correctly executed asynchronously
with respect to (certain) other parts of the program
}

\definition{concurrent program}{
program that uses multiple concurrent interacting threads of execution,
each with its own progress requirements
}

\begin{example}
A program that has separate server and client threads
is a concurrent program.
\end{example}

\definition{parallel program}{
program whose computation is divided into tasks,
which may be distributed across multiple computational units
to be executed simultaneously
}

\begin{note}
If sufficient computational resources are available,
a parallel program may execute significantly faster than
an otherwise equivalent serial program.
\end{note}

\end{definitions}


\clause{Document conventions}

\pnum
\begin{cpp}
Text that is specific to C++
is enclosed in square brackets
and presented in oblique sans-serif type.
\end{cpp}

\pnum
Definitions of terms and grammar non-terminals defined in the C
\begin{cpp}
or C++
\end{cpp}
standard are not duplicated in this document.
Terms and grammar non-terminals defined in this document
are referenced in the index.

\pnum
According to the ISO editing directives,
the use of footnotes
``shall be kept to a minimum.''
Almost all of the footnotes in this document
are not intended to survive to final publication.
Most footnotes are classified by an abbreviation:
\begin{description}
\item[EN]
Editor's note.
These mostly call attention to an area that needs more work.
\item[DFEP]
Departure from existing practice.
\end{description}

\pnum
Annex A contains information
concerning the editing of the LaTeX source of this document.
It will not survive to final publication.

\clause{Counted loops}
\sclause{Introduction}
\pnum
A
\defn{counted loop}
is a
\tcode{for}
statement
\begin{cpp}
or range-based
\tcode{for}
statement
\end{cpp}
that is required to satisfy additional constraints.
The purpose of these constraints is to ensure that
the loop's iteration count can be computed
before the loop body is executed.

\pnum
There shall be no
\tcode{return},
\tcode{break},
\tcode{goto}
or
\tcode{switch}
statement that might transfer control into or out of a counted loop.

\sclause{Constraints on a counted \tcode{for} statement}
\ssclause{Introduction}

\pnum
The syntax of a
\tcode{for}
statement includes three
\defn{control clauses}
between parentheses,
separated by semicolons.
The first of these is called the initialization clause;
the second is called the condition clause or controlling expression;
the third is called the
\defn{loop-increment}.

\pnum
When a constraint limits the form of an expression,
parentheses are allowed around the expression
or any required subexpression.

\ssclause{Constraints on the form of the control clauses}

\pnum
\begin{cpp}
The
\nonterminal{condition}
shall be an expression.

\begin{note}
A condition with declaration form is useful
in a context where a value carries more information
than just whether it is zero or nonzero.
This is not believed to be useful in a counted loop.
\end{note}
\end{cpp}

\pnum
The controlling expression shall be a comparison expression
with one of the following forms:%
\footnote{DFEP:
OpenMP does not (yet) allow comparison with
\tcode{!=}.
}

\begin{bnf}
\br
relational-expression \terminal{<} shift-expression
\br
relational-expression \terminal{>} shift-expression
\br
relational-expression \terminal{<=} shift-expression
\br
relational-expression \terminal{>=} shift-expression
\br
equality-expression \terminal{!=} relational-expression
\end{bnf}

\pnum
Exactly one of the operands of the comparison operator
shall be an identifier designating an induction variable,
as described below.
This induction variable is known as the
\defn{control variable}.
The operand that is not the control variable is called the
\defn{limit expression}.
\begin{cpp}
Any implicit conversion applied to that operand is not considered
part of the limit expression.
\end{cpp}

\pnum
The loop-increment shall be an expression with the following form:%
\footnote{DFEP:
OpenMP and ``classic'' Cilk allow only a single induction variable:
the loop control variable.
Allowing multiple induction variables is implemented in Intel's compiler.
}

\begin{bnf}
\nontermdef{loop-increment}
\br
single-increment
\br
loop-increment \terminal{,} single-increment
\end{bnf}

\begin{bnf}
\nontermdef{single-increment}
\br
identifier \terminal{++}
\br
identifier \terminal{--}
\br
\terminal{++} identifier
\br
\terminal{--} identifier
\br
identifier \terminal{+=} initializer-clause
\br
identifier \terminal{-=} initializer-clause
\br
identifier \terminal{=} identifier \terminal{+} multiplicative-expression
\br
identifier \terminal{=} identifier \terminal{-} multiplicative-expression
\br
identifier \terminal{=} additive-expression \terminal{+} identifier
\end{bnf}

\pnum
\begin{cpp}
Each comma in the grammar of loop-increment shall represent
a use of the built-in comma operator.
\end{cpp}
The identifier in each grammatical alternative for single-increment
names an
\defn{induction variable}.
If
\grammarterm{identifier}
occurs twice in a grammatical alternative for
\grammarterm{single-increment},
the same variable shall be named by both occurrences.
If a grammatical alternative for
\grammarterm{single-increment}
contains a subexpression
that is not an identifier for the induction variable,
that is called the
\defn{stride expression}
for that induction variable.

\pnum
An induction variable shall not be designated by more than one
\grammarterm{single-increment}.

\begin{note}
The control variable is identified by considering
the loop's condition and loop-increment together.
If exactly one operand of the condition comparison is a variable,
it is the control variable, and must be incremented.
If both operands of the condition comparison are variables,
only one is allowed to be incremented;
that one is the control variable.
It is an error if neither operand of the condition comparison is a variable.
\end{note}

\begin{note}
There is no additional constraint on the form
of the initialization clause of a counted \tcode{for} loop.%
\footnote{DFEP:
OpenMP and ``classic'' Cilk require that the control variable be initialized.
This relaxation is implemented in Intel's compiler.
}
\end{note}

\ssclause{Other statically checkable constraints}

\pnum
Each induction variable shall have unqualified integer%
\yescpp{,}
\begin{cpp}
enumeration,
copy-constructible class,
\end{cpp}
or pointer type,
and shall have automatic storage duration.

\pnum
Each stride expression shall have integer
\begin{cpp}
or enumeration
\end{cpp}
type.

\newcommand{\MYin}{\hspace{0.1in}}
\newcommand{\MYcolA}{0.7in}
\newcommand{\MYcolB}{1.0in}
\newcommand{\MYcolC}{1.3in}

\begin{table}[ht]
\caption{%
Method of computing the iteration count
}
\label{tab:itcount}
\centering
\begin{tabular}{|l|l|l|l|l|}
\hline
\parbox[c][30pt]{\MYcolA}{
\bfseries
Form of\\condition
}&
\multicolumn{4}{c|}{
\bfseries
Form of single-increment
}
\\ \hline &
\parbox{\MYcolB}{
\bfseries
\grammarterm{id} \tcode{++}\\
\tcode{++} \grammarterm{id}
}&
\parbox{\MYcolB}{
\bfseries
\grammarterm{id} \tcode{--}\\
\tcode{--} \grammarterm{id}
}&
\parbox[c][40pt]{\MYcolC}{
\bfseries
\grammarterm{id} \tcode{+=} \grammarterm{stride}\\
\grammarterm{id} \tcode{=} \grammarterm{id} \tcode{+} \grammarterm{stride}\\
\grammarterm{id} \tcode{=} \grammarterm{stride}\tcode{+} \grammarterm{id}
}&
\parbox{\MYcolC}{
\bfseries
\grammarterm{id} \tcode{-=} \grammarterm{stride}\\
\grammarterm{id} \tcode{=} \grammarterm{id} \tcode{-} \grammarterm{stride}
}
\\ \hline
\parbox[c][30pt]{\MYcolA}{
\bfseries
\grammarterm{id} \tcode{<} \grammarterm{lim}\\
\grammarterm{lim} \tcode{>} \grammarterm{id}
}&
$((lim)-(id))$&
ERROR&
\parbox{\MYcolC}{
$((lim)-(id)-1)/$\\
\MYin$(stride)+1$
}&
\parbox{\MYcolC}{
$((lim)-(id)-1)/$\\
\MYin$(stride)+1$
}
\\ \hline
\parbox[c][30pt]{\MYcolA}{
\bfseries
\grammarterm{id} \tcode{>} \grammarterm{lim}\\
\grammarterm{lim} \tcode{<} \grammarterm{id}
}&
ERROR&
$((id)-(lim))$&
\parbox{\MYcolC}{
$((id)-(lim)-1)/$\\
\MYin$-(stride)+1$
}&
\parbox{\MYcolC}{
$((id)-(lim)-1)/$\\
\MYin$-(stride)+1$
}
\\ \hline
\parbox[c][30pt]{\MYcolA}{
\bfseries
\grammarterm{id} \tcode{<=} \grammarterm{lim}\\
\grammarterm{lim} \tcode{>=} \grammarterm{id}
}&
\parbox{1in}{
$((lim)-(id))$\\
\MYin+1
}&
ERROR&
\parbox{\MYcolC}{
$((lim)-(id))/$\\
\MYin$(stride)+1$
}&
\parbox{\MYcolC}{
$((lim)-(id))/$\\
\MYin$(stride)+1$
}
\\ \hline
\parbox[c][30pt]{\MYcolA}{
\bfseries
\grammarterm{id} \tcode{>=} \grammarterm{lim}\\
\grammarterm{lim} \tcode{<=} \grammarterm{id}
}&
ERROR&
\parbox{\MYcolB}{
$((id)-(lim))$\\
\MYin$+1$
}&
\parbox{\MYcolC}{
$((id)-(lim))/$\\
\MYin$-(stride)+1$
}&
\parbox{\MYcolC}{
$((id)-(lim))/$\\
\MYin$-(stride)+1$
}
\\ \hline
\parbox{\MYcolA}{
\bfseries
\grammarterm{id} \tcode{!=} \grammarterm{lim}\\
\grammarterm{lim} \tcode{!=} \grammarterm{id}
}&
$((lim)-(id)$&
$((id)-(lim))$&
\parbox[c][72pt]{\MYcolC}{
$((stride)<0)$ \tcode{?}\\
\MYin$((id)-(lim)-1)/$\\
\MYin\MYin$-(stride)+1$ \tcode{:}\\
\MYin$((lim)-(id)-1)/$\\
\MYin\MYin$(stride)+1$
}&
\parbox{\MYcolC}{
$((stride)<0)$ \tcode{?}\\
\MYin$((lim)-(id)-1)/$\\
\MYin\MYin$-(stride)+1$ \tcode{:}\\
\MYin$((id)-(lim)-1)/$\\
\MYin\MYin$(stride)+1$
}
\\ \hline
\end{tabular}
{\bfseries
Legend:
}

\begin{tabular}{|l|l|l|}
\hline
\bfseries Name&
\bfseries In the form of an expression&
\bfseries In the iteration count expression
\\ \hline
$id$&
The name of the control variable.&
\parbox[c][30pt]{2.5in}{
An expression with the type and value\\
of the control variable.
}
\\ \hline
$lim$&
The limit expression.&
\parbox[c][30pt]{2.5in}{
An expression with the type and value\\
of the limit expression.
}
\\ \hline
$stride$&
The stride expression.&
\parbox[c][40pt]{2.5in}{
An expression with the type and value\\
of the stride expression
for the control variable.
}
\\ \hline
\end{tabular}
\end{table}

\pnum
The
\defn{iteration count}
is computed according to
\tref{tab:itcount}.
If the controlling expression uses a relational operator,
and is true when the value of the control variable
is less than (respectively, greater than)
the value of the limit expression,
then the operator in the single-increment for the control variable
shall not be
\tcode{--}
(respectively,
\tcode{++}).
The iteration count is computed after the loop initialization is performed,
and before the control variable is modified by the loop.
\begin{cpp}
The iteration count expression shall be well-formed.
\end{cpp}

\pnum
The type of the difference between the limit expression and the control variable
is the
\defn{subtraction type}\nocpp{.}%
\yescpp{,}
\begin{cpp}
which shall be integral.
When the condition operation is !=,
(limit)-(var) and (var)-(limit) shall have the same type.
\end{cpp}
Each stride expression shall be convertible to the subtraction type.
\begin{cpp}
The loop odr-uses whatever operator-functions are selected
to compute these differences.
\end{cpp}

\begin{cpp}

\begin{table}[ht]
\caption{%
Method of advancing an induction variable
}
\label{tab:inc}
\centering
\begin{tabular}{|l|l|}
\hline
\bfseries Single-increment operator&
\bfseries Expression
\\ \hline
\tcode{++ += +}&
$V$ \tcode{+=} $X$
\\ \hline
\tcode{-- -= -}&
$V$ \tcode{-=} $X$
\\ \hline
\end{tabular}
\end{table}

\pnum
For each induction variable
$V$,
one of the expressions from
\tref{tab:inc}
shall be well-formed,
depending on the operator used in its single-increment.
In the table,
$X$
stands for some expression with the same type as the subtraction type.
The loop odr-uses whatever
\tcode{operator+=}
and
\tcode{operator-=}
functions are selected by these expressions.
\end{cpp}

\ssclause{Dynamic constraints}

\pnum
If an induction variable is modified within the loop
other than as the side effect of its single-increment operation,
the behavior of the program is undefined.%

\begin{cpp}
If evaluation of the iteration count,
or a call to a required
\tcode{operator+=}
or
\tcode{operator-=}
function,
terminates with an exception,
the behavior of the program is undefined.
\end{cpp}

\pnum
If $X$ and $Y$ are values of the control variable
that occur in consecutive evaluations of the loop condition
in the serialization,
then the behavior is undefined if
$((limit) - X) - ((limit) - Y)$,
evaluated in infinite integer precision,
does not equal the stride.

\begin{note}
In other words, the control variable must obey the rules of normal arithmetic.
Unsigned wraparound is not allowed.
\end{note}

\pnum
If the condition expression is true on entry to the loop,
then the behavior is undefined
if the computed iteration count is not greater than zero.
If the computed iteration count is not representable
as a value of type
\tcode{unsigned long long},
the behavior is undefined.

\ssclause{Evaluation relaxations}

\pnum
The stride expressions shall not be evaluated if the iteration count is zero;
otherwise,
the stride and limit expressions are evaluated exactly once.%
\footnote{DFEP:
Neither OpenMP nor Cilk specifies
how many times these expressions must be evaluated.
}
%If execution of a loop iteration alters the value
%of the increment or limit expression,
%the behavior is undefined.

\pnum
Within each iteration of the loop body,
the name of each induction variable refers to a local object,
as if the name were declared as an object within the body of the loop,
with automatic storage duration and with the type of the original object.
\begin{cpp}
If the loop body throws an exception
that is not caught within the same iteration of the loop,
the behavior is undefined, unless otherwise specified.
\end{cpp}

\begin{cpp}
\sclause{Constraints on a counted range-based \tcode{for} statement}
\pnum
In a counted range-based
\tcode{for}
statement ([stmt.ranged] 6.5.4),
the type of the
\tcode{__begin}
variable,
as determined from the
\nonterminal{begin-expr},
shall satisfy the requirements of a random access iterator.
\begin{note}
Intel has not yet implemented support for
a parallel range-based
\tcode{for}
statement.
\end{note}
\end{cpp}


\clause{Parallel loops}

\pnum
A
\defn{parallel loop}
is a
\tcode{for}
statement with loop qualifiers.
The grammar of the iteration statement (6.8.5, paragraph 1)
is modified to read:

\begin{bnf}
\nontermdef{iteration-statement}
\br
\terminal{while} \terminal{(} expression \terminal{)} statement
\br
\terminal{do} statement \terminal{while} \terminal{(} expression \terminal{)} \terminal{;}
\br
\terminal{for} loop-qualifiers\opt \terminal{(}
expression\opt \terminal{;}
expression\opt \terminal{;}
expression\opt \terminal{)} statement
\br
\terminal{for} loop-qualifiers\opt \terminal{(}
declaration
expression\opt \terminal{;}
expression\opt \terminal{)} statement
\end{bnf}

\begin{cpp}
The grammar of
\nonterminal{iteration-statement}
(6.5 [stmt.iter], paragraph 1)
is modified to read:

\begin{bnf}
\nontermdef{iteration-statement}
\br
\terminal{while} \terminal{(} expression \terminal{)} statement
\br
\terminal{do} statement \terminal{while} \terminal{(} expression \terminal{)} \terminal{;}
\br
\terminal{for} loop-qualifiers\opt \terminal{(}
for-init-statement
condition\opt \terminal{;}
expression\opt \terminal{)} statement
\br
\terminal{for} loop-qualifiers\opt \terminal{(}
for-range-declaration \terminal{:}
for-range-initializer \terminal{)} statement
\end{bnf}

\end{cpp}

\pnum
The following rules are added to the grammar:

\begin{bnf}
\nontermdef{loop-qualifiers}
\br
\terminal{_Task_parallel} loop-parameters\opt
\end{bnf}

\begin{bnf}
\nontermdef{loop-parameters}
\br
\terminal{[} expression \terminal{]}
\end{bnf}

\pnum
A parallel loop is a counted loop,
and shall satisfy all the constraints of a counted loop.

\pnum
In a parallel loop with the
\tcode{_Task_parallel}
loop qualifier,
each iteration is a separate task,
which is unsequenced with respect to
all other iterations of that execution of the loop.

\pnum
If loop parameters are specified as part of the loop qualifiers,
the contained expression shall have type
\tcode{cplex_loop_params_t},%
\footnote{EN:
This should probably be
``pointer to \tcode{cplex_loop_params_t}'',
for consistency with the first argument of the set/get macros.
}
as defined in header
\tcode{<cplex.h>}.%
\footnote{DFEP:
This syntax for specifying tuning parameters for a loop
is a CPLEX invention.
}

\pnum
The
\defn{serialization}
of a parallel loop
is obtained by deleting the loop qualifiers from the loop.

\clause{Task statements}
\sclause{Introduction}

\pnum
The grammar of a statement (6.8, paragraph 1)
\begin{cpp}
(clause 6, paragraph 1)
\end{cpp}
is modified to add task-statement as a new alternative.

\ssclause*{Syntax}

\begin{bnf}
\nontermdef{task-statement}
\br
task-block-statement
\br
task-spawn-statement
\br
task-sync-statement
\br
task-call-statement
\end{bnf}

\sclause{The task block statement}
\ssclause*{Syntax}

\begin{bnf}
\nontermdef{task-block-statement}
\br
\terminal{_Task_parallel} \terminal{_Block} reduction-capture\opt{} compound-statement
\end{bnf}

\ssclause*{Constraints}

\pnum
There shall be no
\tcode{switch}
or jump statement that might transfer control into or out of
a task block statement.
\footnote{TODO:
Should task block and/or task spawn be added to the list of statements
terminated by
\tcode{break}?
}

\ssclause*{Semantics}

\pnum
Defines a task block, within which tasks can be spawned.
At the end of the contained compound statement,
execution joins with
all child tasks spawned directly or indirectly
within the compound statement.
\footnote{TODO:
What about C++ EH?
}

\pnum
For a given statement, the
\defn{associated task block}
is defined as follows.
For a statement within a task spawn statement,
there is no associated task block,
except within a nested task block statement
or parallel loop.
For a statement within a task block statement
or parallel loop,
the associated task block is the smallest enclosing task block statement
or parallel loop.
Otherwise, for a statement within the body of a function
declared with the spawning function specifier,
the associated task block is the same as it was
at the point of the task spawning call statement
that invoked the spawning function.
For a statement in any other context,
there is no associated task block.

\begin{note}
Task blocks can be nested lexically and/or dynamically.
Determination of the associated task block is a hybrid process:
lexically within a function,
and dynamically across calls to spawning functions.%
\footnote{DFEP:
In Cilk, this determination can be done entirely lexically.
In OpenMP, this determination can be done entirely dynamically.
}
Code designated for execution in another thread
by means other than a task statement
(e.g. using
\tcode{thrd_create})
is not part of any task block.
\end{note}

\sclause{The task spawn statement}
\ssclause*{Syntax}

\begin{bnf}
\nontermdef{task-spawn-statement}
\br
\terminal{_Task_parallel} \terminal{_Spawn} spawn-capture\opt{} compound-statement
\end{bnf}

\ssclause*{Constraints}

\pnum
A task spawn statement shall have an associated task block.

\pnum
There shall be no
\tcode{switch}
or jump statement that might transfer control into or out of
a task spawn statement.
\footnote{TODO:
Exiting a task spawn or task block with
\tcode{longjmp}
should be undefined behavior.
}

\ssclause*{Semantics}
\pnum
The contained compound statement is executed as a task,
independent of the continued execution
of the associated task block.

\sclause{The task sync statement}
\ssclause*{Syntax}

\begin{bnf}
\nontermdef{task-sync-statement}
\br
\terminal{_Task_parallel} \terminal{_Sync} \terminal{;}
\end{bnf}

\ssclause*{Constraints}

\pnum
A task sync statement shall have an associated task block.

\ssclause*{Semantics}
\pnum
Execution joins with
all child tasks of the associated task block
of the task sync statement.

\sclause{The task spawning call statement}
\ssclause*{Syntax}
\begin{bnf}
\nontermdef{task-call-statement}
\br
\terminal{_Task_parallel} \terminal{_Call} expression-statement
\end{bnf}
\ssclause*{Constraints}

\pnum
A task spawning call statement shall have an associated task block.
\footnote{TODO:
Using the same keyword pair
for the statement prefix and the function specifier
introduces a case where three tokens of lookahead
(and hopefully no more)
are sometimes needed
to disambiguate a declaration from a statement.
Is this what we want?
}

\ssclause*{Semantics}

\pnum
The contained expression statement is executed normally.
Any called spawning function is allowed to spawn tasks;
any such tasks are associated with the associated task block
of the task spawning call statement,
and are
independent of
the statements of the task block
following the task spawning call statement.

\begin{note}
A call to a task spawning function need not be
the ``outermost'' operation of the expression statement.
A task spawning call statement
might invoke more than one spawning function,
or might invoke none.
\end{note}

\sclause{The spawning function specifier}
\ssclause*{Syntax}
\pnum
A new alternative is added to the grammar of function specifier
(6.7.4 paragraph 1):

\begin{bnf}
\nontermdef{function-specifier}
\br
\terminal{_Task_parallel} \terminal{_Call}
\end{bnf}

\ssclause*{Constraints}
\pnum
If a spawning function specifier appears
on any declaration of a function,
it shall appear on every declaration of that function.
A function declared with a spawning function specifier
shall be called only from a task spawning call statement.
\footnote{TODO:
Is a spawning function specifier part of the function's type?
If so, are function-pointer conversions allowed?
}

\clause{Parallel loop hint parameters \tcode{\textless{}cplex.h\textgreater{}}}

\sclause{Introduction}

\pnum
The header
\tcode{\textless cplex.h\textgreater}
defines several types and several macros.

\pnum
The
\pfxdefn{loop_params_t}
type is a structure type
with an unspecified number of members
for specifying parameters for tuning hints for a parallel loop.
A program whose output
depends on the value specified for any tuning hint parameter
is not considered a correct program.

\begin{note}
There is no guarantee that setting any tuning hint parameter
will improve the performance of the program.
\end{note}

\pnum
The
\pfxdefn{sched_kind_t}
type is an enumerated type
with at least the following enumeration constants,
each with nonzero value:

\begin{ttfamily}
\pfxdefn{sched_static}\\
\pfxdefn{sched_dynamic}\\
\pfxdefn{sched_guided}
\end{ttfamily}

\pnum
The
\pfxdefn{workload_t}
type is an enumerated type
with at least the following enumeration constants,
each with nonzero value:

\begin{ttfamily}
\pfxdefn{workload_balanced}\\
\pfxdefn{workload_unbalanced}
\end{ttfamily}

\pnum
The
\pfxdefn{affinity_t}
type is an enumerated type
with at least the following enumeration constants,
each with nonzero value:

\begin{ttfamily}
\pfxdefn{affinity_close}\\
\pfxdefn{affinity_spread}
\end{ttfamily}

\pnum
When an object of type
\pfx{loop_params_t}
is used as the loop parameter of a parallel loop,
the loop is described as being associated with the object.%
\footnote{EN:
What if the object is modified during the execution of the loop?
}
When executing a parallel loop associated with an object of type
\pfx{loop_params_t},
for any parameter for which the corresponding member has the value zero,
an unspecified default value is used.

\pnum
Each parameter is represented by a pair of macros:
one to set the value of the parameter in the parameter block,
and one to get the value of the parameter from the parameter block.

\begin{note}
Because these methods are specified as macros,
not functions,
taking the address of any of them need not be supported.
However, an implementation is also free
to provide functions with these names.
\end{note}

\begin{example}
Hint parameters for a parallel loop can be specified as follows:
\begin{verbatim}
#include <cplex.h>
#include <stdlib.h>
int main(int argc, char *argv[])
{
    cplex_loop_params_t hints = { 0 };
    if (argc > 1) {
        cplex_set_num_threads(&hints, atoi(argv[1]));
    }
    cplex_set_chunk_size(&hints, 1000);
    for _Task_parallel[hints] (long i = 0; i < 1000000; i++) {
        do_something_with(i);
    }
}
\end{verbatim}
\end{example}

\sclause{The \tcode{num_threads} parameter}
\ssclause*{Synopsis}
\begin{ttfamily}
\#include <cplex.h>\\
void \pfxdefn{set_num_threads}(\pfx{loop_params_t} *hints, int num_threads);\\
int \pfx{get_num_threads}(\pfx{loop_params_t} *hints);
\end{ttfamily}

\ssclause*{Description}
\pnum
The
\pfx{set_num_threads}
macro sets to
\tcode{num_threads}
the recommended number of execution agents to be used
to execute the iterations of a parallel loop
associated with the object pointed to by
\tcode{hints}.

\sclause{The \tcode{chunk_size} parameter}
\ssclause*{Synopsis}
\begin{ttfamily}
\#include <cplex.h>\\
void \pfxdefn{set_chunk_size}(\pfx{loop_params_t} *hints, int chunk_size);\\
int \pfx{get_chunk_size}(\pfx{loop_params_t} *hints);
\end{ttfamily}

\ssclause*{Description}
\pnum
The
\pfx{set_chunk_size}
macro sets to
\tcode{chunk_size}
the recommended maximum number of iterations
of a parallel loop associated with the object pointed to by
\tcode{hints}
to be grouped together to be executed sequentially
in a single thread of execution.

\sclause{The \tcode{schedule_kind} parameter}
\ssclause*{Synopsis}
\begin{ttfamily}
\#include <cplex.h>\\
void \pfxdefn{set_schedule_kind}(\pfx{loop_params_t} *hints, \pfx{sched_kind_t} kind);\\
\pfx{sched_kind_t} \pfx{get_schedule_kind}(\pfx{loop_params_t} *hints);
\end{ttfamily}

\ssclause*{Description}
\pnum
The
\pfx{set_schedule_kind}
macro sets to
\tcode{kind}
the recommended scheduling algorithm
for a parallel loop associated with the object pointed to by
\tcode{hints}.

\begin{note}
Setting the
\tcode{schedule_kind}
parameter to a particular value
may (but need not)
select the corresponding OpenMP loop-scheduling algorithm.
\end{note}

\sclause{The \tcode{workload_balance} parameter}
\ssclause*{Synopsis}
\begin{ttfamily}
\#include <cplex.h>\\
void \pfxdefn{set_workload_balance}(\pfx{loop_params_t} *hints, \pfx{workload_t} kind);\\
\pfx{workload_t} \pfx{get_workload_balance}(\pfx{loop_params_t} *hints);
\end{ttfamily}

\ssclause*{Description}
\pnum
The
\pfx{set_workload_balance}
macro sets to
\tcode{kind}
the workload-balancing characteristic
for a parallel loop associated with the object pointed to by
\tcode{hints}.

\pnum
For a loop with a balanced workload,
each iteration should be assumed to execute
in approximately the same amount of time.
A loop with an unbalanced workload
should be assumed to have iterations
taking widely varying amounts of time.

\begin{note}
This parameter is semantically a statement about the associated loop,
whereas the
\tcode{schedule_kind}
parameter is semantically a request to the implementation.
Setting this parameter to
\pfx{workload_balanced}
may have an effect similar to setting the schedule to
\pfx{schedule_static}.
Setting this parameter to
\pfx{workload_unbalanced}
may have an effect similar to setting the schedule to
\pfx{schedule_dynamic}
or
\pfx{schedule_guided}.
\end{note}

\sclause{The \tcode{affinity} parameter}
\ssclause*{Synopsis}
\begin{ttfamily}
\#include <cplex.h>\\
void \pfxdefn{set_affinity}(\pfx{loop_params_t} *hints, \pfx{affinity_t} kind);\\
\pfx{affinity_t} \pfx{get_affinity}(\pfx{loop_params_t} *hints);
\end{ttfamily}

\ssclause*{Description}
\pnum
The
\pfx{set_affinity}
macro sets to
\tcode{kind}
the recommended affinity
for a parallel loop associated with the object pointed to by
\tcode{hints}.

\pnum
The affinity of a loop indicates whether the loop benefits
from being executed by co-located hardware threads,
or whether performance is likely to improve if the software threads
are spread over multiple cores.


%\%include{samples}

\bibannex
\begin{references}
\reference{}{Intel\copyright{} Cilk\texttrademark{} Plus
Language Extension Specification,}
{Intel Corporation}:
\isourl{https://www.cilkplus.org/sites/default/files/open_specifications/Intel_Cilk_plus_lang_spec_1.2.htm}
\reference{}{OpenMP Application Program Interface,}
{OpenMP Architecture Review Board}:
\isourl{http://www.openmp.org/mp-documents/OpenMP4.0.0.pdf}
\end{references}

\printindex
\end{document}
